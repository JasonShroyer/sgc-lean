% ============================================================================
% SECTION 2: THE KINEMATICS OF FINITE GEOMETRY
% ============================================================================
% Operator-algebraic and geometric foundations for finite-state thermodynamics.
% Sign convention: L is the backward generator (negative semi-definite),
% H = -½(L + L†) is the symmetrization (positive semi-definite).
% ============================================================================

\section{The Kinematics of Finite Geometry}
\label{sec:kinematics}

This section rigorously defines the operator-algebraic and geometric foundations of the framework. We introduce the causal generator for continuous-time dynamics on a finite directed graph, the weighted $L^2(\pi)$ geometry, and the associated Dirichlet form. Particular emphasis is placed on non-reversible chains, where the generator is non-self-adjoint; we employ additive symmetrization to obtain a self-adjoint operator governing energy dissipation and the variational spectral gap.

% ----------------------------------------------------------------------------
\subsection{The Causal Generator}
\label{sec:kinematics:generator}
% ----------------------------------------------------------------------------

The system dynamics are modeled as a continuous-time Markov chain on a finite directed graph.

\begin{definition}[Weighted Graph and Generator]
\label{def:weighted-graph-generator}
\leanlink{UPAT/Axioms/Geometry.lean}

Let $V$ be a finite set of states. The causal structure is specified by a weighted adjacency matrix $W : V \times V \to \mathbb{R}_{\geq 0}$ with $W_{ii} = 0$. The out-degree is $d_i \coloneqq \sum_{j} W_{ij}$. The \textbf{backward generator} $L$ acts on observables $f : V \to \mathbb{R}$ as
\begin{equation}
(Lf)(i) \coloneqq \sum_{j} W_{ij}\bigl(f(j) - f(i)\bigr).
\end{equation}
In matrix notation, $L = W - D$, where $D = \mathrm{diag}(d_i)$ is the diagonal out-degree matrix. This operator drives the forward evolution of probabilities via $\dot{\mu} = \mu L$ and the backward evolution of observables via $\dot{f} = Lf$.
\end{definition}

\begin{assumption}[Ergodicity]
\label{assum:ergodicity}
The graph is strongly connected with sufficient positive rates to ensure a unique strictly positive stationary measure $\pi : V \to \mathbb{R}_{>0}$ satisfying $\pi L = 0$~\cite{LevinPeres2017,Montenegro2006}.
\end{assumption}

Discrete-time analogs replace $L$ with $P - I$, where $P_{ij} \coloneqq W_{ij}/d_i$ for $d_i > 0$; the subsequent analysis adapts straightforwardly.

% ----------------------------------------------------------------------------
\subsection{The $L^2(\pi)$ Geometry and Dirichlet Form}
\label{sec:kinematics:geometry}
% ----------------------------------------------------------------------------

The stationary measure induces a weighted Hilbert structure.

\begin{definition}[$L^2(\pi)$ Inner Product]
\label{def:L2pi-inner-product}
\leanlink{UPAT/Stability/Core/Assumptions.lean}

The space of observables forms a Hilbert space with inner product
\begin{equation}
\langle u, v \rangle_\pi \coloneqq \sum_{x \in V} \pi(x)\, u(x)\, v(x)
\end{equation}
and norm $\|u\|_\pi \coloneqq \sqrt{\langle u, u \rangle_\pi}$.
\end{definition}

Under detailed balance ($\pi_i W_{ij} = \pi_j W_{ji}$), the generator $L$ is self-adjoint and negative semi-definite on $L^2(\pi)$~\cite{Chung1997}. In general non-reversible cases, $L$ is non-self-adjoint.

\begin{definition}[$\pi$-Adjoint and Additive Symmetrization]
\label{def:pi-adjoint-symmetrization}
\leanlink{UPAT/Stability/Defs.lean}

The $\pi$-adjoint $L^\dagger$ is defined by $\langle Lu, v \rangle_\pi = \langle u, L^\dagger v \rangle_\pi$, with explicit action
\begin{equation}
(L^\dagger f)(i) = \sum_{j} \frac{\pi_j}{\pi_i} W_{ji}\bigl(f(j) - f(i)\bigr).
\end{equation}
The \textbf{additive symmetrization} is $H \coloneqq -\tfrac{1}{2}(L + L^\dagger)$, a self-adjoint positive semi-definite operator on $L^2(\pi)$~\cite{Choi2023,Dechant2020}.
\end{definition}

\begin{definition}[Dirichlet Form]
\label{def:dirichlet-form}
\leanlink{UPAT/Stability/Functoriality/Lumpability.lean}

The \textbf{Dirichlet form} is
\begin{equation}
\mathcal{E}(u) \coloneqq -\langle u, Lu \rangle_\pi = \langle u, Hu \rangle_\pi.
\end{equation}
This equals the symmetric expression
\begin{equation}
\mathcal{E}(u) = \frac{1}{2} \sum_{i,j} \pi_i W_{ij}\bigl(u(j) - u(i)\bigr)^2 \cdot \frac{\pi_i W_{ij} + \pi_j W_{ji}}{2\pi_i W_{ij}},
\end{equation}
which is non-negative definite and quantifies gradient energy along directed edges~\cite{Gaudilliere2014,Mielke2011}. In non-reversible settings, $H$ provides robust variational bounds despite asymmetry.
\end{definition}

% ----------------------------------------------------------------------------
\subsection{The Spectral Gap}
\label{sec:kinematics:spectral-gap}
% ----------------------------------------------------------------------------

Asymptotic consolidation is governed by the spectral gap of the symmetrized dynamics.

\begin{definition}[Variational Spectral Gap]
\label{def:spectral-gap-variational}
\leanlink{UPAT/Stability/Functoriality/Lumpability.lean}

The \textbf{spectral gap} $\lambda_{\mathrm{gap}}$ is
\begin{equation}
\lambda_{\mathrm{gap}} \coloneqq \inf_{u \perp_\pi \mathbf{1},\, u \neq 0} \frac{\mathcal{E}(u)}{\|u\|_\pi^2} = \inf_{u \perp_\pi \mathbf{1},\, u \neq 0} \frac{\langle u, Hu \rangle_\pi}{\langle u, u \rangle_\pi},
\end{equation}
where the infimum is over functions normalized and orthogonal to constants in $L^2(\pi)$.
\end{definition}

Under ergodicity, $\lambda_{\mathrm{gap}} > 0$~\cite{Diaconis1996}. For non-self-adjoint $L$, eigenvalues may be complex; $\lambda_{\mathrm{gap}}$ lower-bounds the real part of non-zero eigenvalues, ensuring exponential variance decay (Poincaré inequality)~\cite{Chatterjee2025spectral,Dechant2025finite}. Singular-value characterizations extend this to precise mixing-time control in non-reversible chains~\cite{Chatterjee2025definition}. A finite $\lambda_{\mathrm{gap}}$ prohibits perpetual superdiffusion, consistent with inverse thermodynamic uncertainty relations~\cite{Vo2025inverse}.

\textbf{Methodological Note.} Operator definitions, adjoint relations, and spectral properties (including complex cases for non-normal matrices) are formally verified in the Lean 4 modules \texttt{UPAT.Axioms.Geometry} and \texttt{UPAT.Stability.Defs}.
\leanlink{UPAT/Stability/Defs.lean}

% ----------------------------------------------------------------------------
\subsection{Heat Kernel Observables}
% ----------------------------------------------------------------------------

\begin{definition}[Heat Kernel]
\label{def:heat-kernel}
\leanlink{UPAT/Stability/Envelope.lean}

The \textbf{heat kernel} (heat semigroup) is:
\[
K(t) \coloneqq e^{tL}
\]
It satisfies the heat equation $\frac{d}{dt}K(t) = L \cdot K(t)$ with $K(0) = I$.
\end{definition}

\begin{definition}[Normalized Return Probability]
\label{def:K-norm}
\leanlink{UPAT/Stability/Defs.lean}

The \textbf{normalized return probability observable}:
\[
\widetilde{K}(t, x) \coloneqq 1 - \frac{K(t)_{xx}}{\pi_x}
\]
\end{definition}

\begin{definition}[Expected Log-Observable]
\label{def:E-log-K-norm}
\leanlink{UPAT/Stability/Defs.lean}

The \textbf{expected log-observable}:
\[
\mathbb{E}_\pi[\log \widetilde{K}](t) \coloneqq \sum_{x \in V} \pi_x \cdot \log\left( \widetilde{K}(t, x) + \varepsilon \right)
\]
where $\varepsilon > 0$ is a regularization parameter.
\end{definition}

\begin{definition}[Stability Flow]
\label{def:beta-t}
\leanlink{UPAT/Stability/Defs.lean}

The \textbf{stability flow} is the time derivative of the expected log-observable:
\[
\beta(t) \coloneqq \frac{d}{dt} \mathbb{E}_\pi[\log \widetilde{K}](t)
\]
\end{definition}

The differentiability of the heat kernel and normalized return probability is guaranteed by standard spectral theory (details in Appendix~\ref{sec:aux:heat-kernel}).

% ----------------------------------------------------------------------------
\subsection{The Functorial Heat Dominance Theorem}
% ----------------------------------------------------------------------------

\begin{theorem}[Functorial Heat Dominance Theorem (FHDT)]
\label{thm:FHDT}
\leanlink{UPAT/Stability/Defs.lean}

Given:
\begin{itemize}
    \item $V$ nontrivial finite type
    \item $L : V \times V \to \mathbb{R}$ generator with $L \cdot \mathbf{1} = 0$
    \item $H : V \times V \to \mathbb{R}$ self-adjoint, PSD, with $H \cdot \mathbf{1} = 0$
    \item Symmetrization relation: $\langle Lu, v \rangle_\pi + \langle u, Lv \rangle_\pi = -2\langle Hu, v \rangle_\pi$
    \item $\gamma \coloneqq \mathrm{SpectralGap}_\pi(H) > 0$
    \item $\pi : V \to \mathbb{R}^+$ stationary with $\sum_x \pi_x = 1$
    \item $\varepsilon > 0$ with $\widetilde{K}(t,x) + \varepsilon > 0$ for all $x, t$
\end{itemize}

Then there exists $C \ge 0$ such that for all $t \ge 0$:
\[
|\beta(t)| \le C \cdot e^{-\gamma t}
\]

\begin{proof}
\textbf{Step 1} (Derivative Formula):
\[
\beta(t) = \sum_{x} \pi_x \cdot \frac{\frac{d}{dt}\widetilde{K}(t,x)}{\widetilde{K}(t,x) + \varepsilon}
= -\sum_x \frac{(L \cdot K(t))_{xx}}{\widetilde{K}(t,x) + \varepsilon}
\]

\textbf{Step 2} (Denominator Bound): 
Given $\varepsilon_{\min} > 0$ with $\widetilde{K} + \varepsilon \ge \varepsilon_{\min}$:
\[
|\beta(t)| \le \frac{1}{\varepsilon_{\min}} \sum_x |(L \cdot K(t))_{xx}|
\]

\textbf{Step 3} (Diagonal Bound via Pillar 3):
By \texttt{sum\_abs\_diag\_le\_card\_opNorm}:
\[
\sum_x |A_{xx}| \le |V| \cdot \|A\|_{\mathrm{op},\pi}
\]

\textbf{Step 4} (Operator Factorization):
$L \cdot K(t)$ factors through $P_\perp$ since $L$ kills constants and $K(t)$ preserves them:
\[
L \cdot K(t) = L \circ K(t) \circ P_\perp
\]

\textbf{Step 5} (Envelope Bound via Pillar 2):
By \texttt{sector\_envelope\_bound\_canonical}:
\[
\|K(t) \circ P_\perp\|_{\mathrm{op},\pi} \le e^{-\gamma t}
\]

\textbf{Step 6} (Submultiplicativity):
By \texttt{opNorm\_pi\_comp}:
\[
\|L \cdot K(t)\|_{\mathrm{op},\pi} \le \|L\|_{\mathrm{op},\pi} \cdot e^{-\gamma t}
\]

\textbf{Step 7} (Final Assembly):
\[
|\beta(t)| \le \frac{|V| \cdot \|L\|_{\mathrm{op},\pi}}{\varepsilon_{\min} \cdot \pi_{\min}} \cdot e^{-\gamma t}
\eqqcolon C \cdot e^{-\gamma t}
\]
\end{proof}
\end{theorem}

% ============================================================================
% END KINEMATICS
% ============================================================================
