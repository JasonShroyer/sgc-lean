% ============================================================================
% SCALING: Renormalization and Lumpability
% ============================================================================
% Literal translation from:
%   - src/UPAT/Stability/Functoriality/Lumpability.lean
% ============================================================================

\section{Scaling: Renormalization Group}
\label{sec:scaling}

% ----------------------------------------------------------------------------
\subsection{Partition Structure}
\label{sec:scaling:partition}
% ----------------------------------------------------------------------------

\leanlink{UPAT/Stability/Functoriality/Lumpability.lean}

The renormalization group acts on the state space via \textbf{partitions}. A partition $\mathcal{P}$ of $V$ is an equivalence relation $\sim$ grouping microstates into macroscopic blocks. The \textbf{quotient space} $\bar{V} \coloneqq V/{\sim}$ represents the coarse-grained state space, with quotient map $q : V \to \bar{V}$ given by $q(x) = [x]_\sim$. The stationary measure aggregates to the \textbf{quotient distribution}:
\[
\bar{\pi}(\bar{a}) \coloneqq \sum_{x \in \bar{a}} \pi_x
\]
which inherits positivity and normalization from $\pi$.

% ----------------------------------------------------------------------------
\subsection{Strong Lumpability}
% ----------------------------------------------------------------------------

\begin{definition}[Strong Lumpability]
\label{def:strongly-lumpable}
\leanlink{UPAT/Stability/Functoriality/Lumpability.lean}

A generator $L$ is \textbf{strongly lumpable} with respect to partition $P$ iff:
\[
\forall x, y \in V,\; x \sim y \implies \forall \bar{b} \in \bar{V},\;
\sum_{z \in \bar{b}} L_{xz} = \sum_{z \in \bar{b}} L_{yz}
\]
This ensures the quotient generator is well-defined.
\end{definition}

\begin{definition}[Row Sum Block]
\label{def:row-sum-block}
\leanlink{UPAT/Stability/Functoriality/Lumpability.lean}

\[
\mathrm{RowSum}(L, i, \bar{B}) \coloneqq \sum_{k \in \bar{B}} L_{ik}
\]
\end{definition}

\begin{lemma}[Row Sum Constant on Classes]
\label{lem:row-sum-const}
\leanlink{UPAT/Stability/Functoriality/Lumpability.lean}

Under strong lumpability, for $i \sim j$:
\[
\mathrm{RowSum}(L, i, \bar{B}) = \mathrm{RowSum}(L, j, \bar{B})
\]
\end{lemma}

% ----------------------------------------------------------------------------
\subsection{The Quotient Generator}
% ----------------------------------------------------------------------------

\begin{definition}[Quotient Generator]
\label{def:quotient-generator}
\leanlink{UPAT/Stability/Functoriality/Lumpability.lean}

The \textbf{quotient generator} $\bar{L} : \bar{V} \times \bar{V} \to \mathbb{R}$:
\[
\bar{L}_{\bar{A}\bar{B}} \coloneqq \sum_{k \in \bar{B}} L_{\mathrm{rep}(\bar{A}), k}
\]
where $\mathrm{rep}(\bar{A})$ is any representative of class $\bar{A}$.
Under strong lumpability, this is independent of the choice of representative.
\end{definition}

% ----------------------------------------------------------------------------
\subsection{The Lift Operator}
\label{sec:scaling:lift}
% ----------------------------------------------------------------------------

\leanlink{UPAT/Stability/Functoriality/Lumpability.lean}

The \textbf{Lift Operator} $K : V \times \bar{V} \to \mathbb{R}$ embeds coarse observables into the fine space. For $f : \bar{V} \to \mathbb{R}$, the lift is $(\mathrm{lift}\, f)(x) \coloneqq f(q(x))$, equivalently $\mathrm{lift}\, f = K \cdot f$ where $K_{i\bar{B}} = \mathbf{1}_{q(i) = \bar{B}}$. A function $f : V \to \mathbb{R}$ is \textbf{block-constant} iff $x \sim y \implies f(x) = f(y)$. The lifted functions are precisely the block-constant functions: $f$ is block-constant if and only if $f = \mathrm{lift}\, g$ for some $g$.

% ----------------------------------------------------------------------------
\subsection{The Intertwining Theorem (RG Flow)}
% ----------------------------------------------------------------------------

\begin{theorem}[Intertwining (Dynkin Formula)]
\label{thm:intertwining}
\leanlink{UPAT/Stability/Functoriality/Lumpability.lean}

Under strong lumpability:
\[
L \cdot K = K \cdot \bar{L}
\]
The original dynamics $L$ and quotient dynamics $\bar{L}$ are related by the lift operator.
This is the fundamental algebraic property of strong lumpability.

\begin{proof}
For each $(i, \bar{B})$:
\begin{align*}
(L \cdot K)_{i\bar{B}} &= \sum_k L_{ik} K_{k\bar{B}} = \sum_{k \in \bar{B}} L_{ik} = \mathrm{RowSum}(L, i, \bar{B}) \\
(K \cdot \bar{L})_{i\bar{B}} &= \sum_{\bar{C}} K_{i\bar{C}} \bar{L}_{\bar{C}\bar{B}} = \bar{L}_{q(i), \bar{B}} = \mathrm{RowSum}(L, \mathrm{rep}(q(i)), \bar{B})
\end{align*}
By Lemma~\ref{lem:row-sum-const}, these are equal since $i \sim \mathrm{rep}(q(i))$.
\end{proof}
\end{theorem}

\begin{theorem}[Power Intertwining]
\label{thm:intertwining-pow}
\leanlink{UPAT/Stability/Functoriality/Lumpability.lean}

For all $n \in \mathbb{N}$:
\[
L^n \cdot K = K \cdot \bar{L}^n
\]

\begin{proof}
By induction on $n$:
\begin{itemize}
    \item \textbf{Base}: $L^0 \cdot K = I \cdot K = K = K \cdot I = K \cdot \bar{L}^0$
    \item \textbf{Step}: $L^{n+1} \cdot K = L^n \cdot L \cdot K = L^n \cdot K \cdot \bar{L} = K \cdot \bar{L}^n \cdot \bar{L} = K \cdot \bar{L}^{n+1}$
\end{itemize}
\end{proof}
\end{theorem}

\begin{theorem}[Vector Intertwining]
\label{thm:L-lift-eq}
\leanlink{UPAT/Stability/Functoriality/Lumpability.lean}

\[
L \cdot (\text{lift}\, f) = \text{lift}\, (\bar{L} \cdot f)
\]
The generator $L$ applied to a lifted function equals the lift of $\bar{L}$ applied to $f$.

\begin{proof}
By matrix intertwining: $L \cdot K \cdot f = K \cdot \bar{L} \cdot f$.
\end{proof}
\end{theorem}

% ----------------------------------------------------------------------------
\subsection{Lift Isometry}
% ----------------------------------------------------------------------------

\begin{theorem}[Lift Isometry]
\label{thm:lift-isometry}
\leanlink{UPAT/Stability/Functoriality/Lumpability.lean}

The lift preserves the weighted $L^2$ inner product:
\[
\langle \text{lift}\, f, \text{lift}\, g \rangle_\pi = \langle f, g \rangle_{\bar{\pi}}
\]

\begin{proof}
Group by equivalence class:
\begin{align*}
\langle \text{lift}\, f, \text{lift}\, g \rangle_\pi 
&= \sum_x \pi_x \cdot f(q(x)) \cdot g(q(x)) \\
&= \sum_{\bar{A}} \left( \sum_{x \in \bar{A}} \pi_x \right) \cdot f(\bar{A}) \cdot g(\bar{A}) \\
&= \sum_{\bar{A}} \bar{\pi}(\bar{A}) \cdot f(\bar{A}) \cdot g(\bar{A}) = \langle f, g \rangle_{\bar{\pi}}
\end{align*}
\end{proof}
\end{theorem}

% ----------------------------------------------------------------------------
\subsection{Dirichlet Form Descent}
% ----------------------------------------------------------------------------

\begin{theorem}[Dirichlet Form Lift Equality]
\label{thm:dirichlet-lift}
\leanlink{UPAT/Stability/Functoriality/Lumpability.lean}

\[
\mathcal{E}(\text{lift}\, f) = \bar{\mathcal{E}}(f)
\]

\begin{proof}
Combine forward quadratic form equality:
\[
\langle \text{lift}\, f, L \cdot \text{lift}\, f \rangle_\pi = \langle f, \bar{L} \cdot f \rangle_{\bar{\pi}}
\]
with backward quadratic form equality (by symmetry of $\langle\cdot,\cdot\rangle$).
\end{proof}
\end{theorem}

\begin{theorem}[Rayleigh Quotient Lift Equality]
\label{thm:rayleigh-lift}
\leanlink{UPAT/Stability/Functoriality/Lumpability.lean}

\[
R(\text{lift}\, f) = \bar{R}(f)
\]

\begin{proof}
Numerator: $\mathcal{E}(\text{lift}\, f) = \bar{\mathcal{E}}(f)$ by Theorem~\ref{thm:dirichlet-lift}.
Denominator: $\|\text{lift}\, f\|^2_\pi = \|f\|^2_{\bar{\pi}}$ by Theorem~\ref{thm:lift-isometry}.
\end{proof}
\end{theorem}

% ----------------------------------------------------------------------------
\subsection{Spectral Gap Monotonicity}
% ----------------------------------------------------------------------------

\begin{theorem}[Spectral Gap Non-Decrease]
\label{thm:gap-non-decrease}
\leanlink{UPAT/Stability/Functoriality/Lumpability.lean}

Under strong lumpability:
\[
\bar{\gamma} \ge \gamma
\]
Coarse-graining \textbf{cannot decrease} the spectral gap.

\begin{proof}
\textbf{Key insight}: 
\begin{align*}
\gamma &= \inf \{ R(u) \mid u \neq 0,\; u \perp_\pi \mathbf{1} \} \tag{all functions} \\
\bar{\gamma} &= \inf \{ R(u) \mid u \text{ block-constant},\; u \neq 0,\; u \perp_\pi \mathbf{1} \} \tag{block-constant only}
\end{align*}
Since block-constant functions form a \textbf{subset}:
\[
\inf(\text{subset}) \ge \inf(\text{total})
\]
By Lemma~\ref{lem:block-constant-iff-lift} and Theorem~\ref{thm:rayleigh-lift}, the block-constant 
Rayleigh set equals the quotient Rayleigh set.
\end{proof}
\end{theorem}

% ----------------------------------------------------------------------------
\subsection{Physical Interpretation: The Arrow of Time}
% ----------------------------------------------------------------------------

\begin{corollary}[Stability is Scale-Invariant]
\label{cor:stability-scale-invariant}

The stability triad $(\gamma, \beta(t), B(t))$ is \textbf{functorial} under coarse-graining:
\begin{enumerate}
    \item \textbf{Intertwining}: Dynamics commutes with lifting ($L \cdot K = K \cdot \bar{L}$)
    \item \textbf{Isometry}: Lift preserves energy ($\mathcal{E}(\text{lift}\, f) = \bar{\mathcal{E}}(f)$)
    \item \textbf{Gap Monotonicity}: Coarse-graining enhances stability ($\bar{\gamma} \ge \gamma$)
\end{enumerate}

The \textbf{Arrow of Time} emerges: coarse-graining is thermodynamically irreversible
because the spectral gap can only increase, accelerating convergence to equilibrium.
\end{corollary}

% ============================================================================
% END SCALING
% ============================================================================
