% formalism.tex
% Literal symbol-for-symbol translation of UPAT Lean 4 proofs
% Generated from src/ — DO NOT EDIT MANUALLY

\usepackage{physics}
\usepackage{amsmath,amssymb,amsthm}
\usepackage{hyperref}

% ============================================================================
% THEOREM ENVIRONMENTS
% ============================================================================
\theoremstyle{definition}
\newtheorem{definition}{Definition}[section]
\newtheorem{structure}[definition]{Structure}

\theoremstyle{plain}
\newtheorem{theorem}[definition]{Theorem}
\newtheorem{lemma}[definition]{Lemma}
\newtheorem{corollary}[definition]{Corollary}

% ============================================================================
% PART I: L²(π) GEOMETRY FOUNDATION
% Source: UPAT/Axioms/Geometry.lean
% ============================================================================

\section{L²(π) Geometry Foundation}
\label{sec:geometry}

\leanlink{UPAT/Axioms/Geometry.lean}

\subsection{Core Definitions}

\begin{definition}[Constant Vector One]
\label{def:constant-vec-one}
\leanlink{UPAT/Axioms/Geometry.lean}

Given $V$ a finite type,
\[
\mathbf{1} : V \to \mathbb{R}, \quad \mathbf{1}(v) \coloneqq 1
\]
\end{definition}

\begin{definition}[Weighted $L^2(\pi)$ Inner Product]
\label{def:inner-pi}
\leanlink{UPAT/Axioms/Geometry.lean}

Given $\pi : V \to \mathbb{R}$ (weight distribution) and $u, v : V \to \mathbb{R}$,
\[
\ip{u}{v}_\pi \coloneqq \sum_{x \in V} \pi(x) \cdot u(x) \cdot v(x)
\]
\end{definition}

\begin{definition}[Weighted Squared Norm]
\label{def:norm-sq-pi}
\leanlink{UPAT/Axioms/Geometry.lean}

\[
\norm{v}^2_\pi \coloneqq \ip{v}{v}_\pi = \sum_{x \in V} \pi(x) \cdot v(x)^2
\]
\end{definition}

\begin{definition}[Weighted Norm]
\label{def:norm-pi}
\leanlink{UPAT/Axioms/Geometry.lean}

\[
\norm{v}_\pi \coloneqq \sqrt{\norm{v}^2_\pi}
\]
\end{definition}

\subsection{Bilinearity Lemmas}

\begin{lemma}[Additivity in First Argument]
\label{lem:inner-pi-add-left}
\leanlink{UPAT/Axioms/Geometry.lean}

\[
\ip{u + v}{w}_\pi = \ip{u}{w}_\pi + \ip{v}{w}_\pi
\]
\begin{proof}
By simplification on $\texttt{Pi.add\_apply}$. We rewrite using $\texttt{Finset.sum\_add\_distrib}$. Therefore, by algebraic manipulation, we obtain the result.
\end{proof}
\end{lemma}

\begin{lemma}[Scalar Multiplication in First Argument]
\label{lem:inner-pi-smul-left}
\leanlink{UPAT/Axioms/Geometry.lean}

\[
\ip{c \cdot u}{v}_\pi = c \cdot \ip{u}{v}_\pi
\]
\begin{proof}
By simplification on $\texttt{Pi.smul\_apply}$ and $\texttt{smul\_eq\_mul}$. We rewrite using $\texttt{Finset.mul\_sum}$. Therefore, by algebraic manipulation, we obtain the result.
\end{proof}
\end{lemma}

\begin{lemma}[Symmetry]
\label{lem:inner-pi-comm}
\leanlink{UPAT/Axioms/Geometry.lean}

\[
\ip{u}{v}_\pi = \ip{v}{u}_\pi
\]
\begin{proof}
By simplification using commutativity of multiplication.
\end{proof}
\end{lemma}

\subsection{Norm Properties}

\begin{lemma}[Norm Squared as Sum]
\label{lem:norm-sq-pi-eq-sum}
\leanlink{UPAT/Axioms/Geometry.lean}

\[
\norm{h}^2_\pi = \sum_{v \in V} \pi(v) \cdot h(v)^2
\]
\begin{proof}
By unfolding $\texttt{norm\_sq\_pi}$ and $\texttt{inner\_pi}$. We have $h(v) \cdot h(v) = h(v)^2$.
\end{proof}
\end{lemma}

\begin{lemma}[Positivity of Squared Norm]
\label{lem:norm-sq-pi-pos}
\leanlink{UPAT/Axioms/Geometry.lean}

Given $\forall v,\, 0 < \pi(v)$ and $u \neq 0$,
\[
0 < \norm{u}^2_\pi
\]
\begin{proof}
We first establish that there exists $v_0$ with $u(v_0) \neq 0$. Given $u \neq 0$, by contradiction and $\texttt{funext}$, such $v_0$ exists. We have $0 < \pi(v_0) \cdot u(v_0)^2$ since $\pi(v_0) > 0$ and $u(v_0)^2 > 0$. By $\texttt{Finset.sum\_pos'}$, the sum of non-negative terms with one positive term is positive.
\end{proof}
\end{lemma}

\begin{lemma}[Zero Iff Zero Function]
\label{lem:norm-sq-pi-eq-zero-iff}
\leanlink{UPAT/Axioms/Geometry.lean}

Given $\forall v,\, 0 < \pi(v)$,
\[
\norm{h}^2_\pi = 0 \iff \forall v,\, h(v) = 0
\]
\begin{proof}
($\Rightarrow$) By $\texttt{Finset.sum\_eq\_zero\_iff\_of\_nonneg}$, each term $\pi(v) \cdot h(v)^2 = 0$. Since $\pi(v) > 0$, we have $h(v)^2 = 0$, hence $h(v) = 0$.

($\Leftarrow$) If $h = 0$, then each term is zero.
\end{proof}
\end{lemma}

\subsection{Cauchy-Schwarz Inequality}

\begin{lemma}[Cauchy-Schwarz for $L^2(\pi)$]
\label{lem:cauchy-schwarz-pi}
\leanlink{UPAT/Axioms/Geometry.lean}

Given $\forall v,\, 0 < \pi(v)$,
\[
\abs{\ip{f}{g}_\pi} \le \norm{f}_\pi \cdot \norm{g}_\pi
\]
\begin{proof}
Let $a \coloneqq \norm{g}^2_\pi$, $b \coloneqq \ip{f}{g}_\pi$, $c \coloneqq \norm{f}^2_\pi$.

Define $P(t) \coloneqq \norm{f + t \cdot g}^2_\pi$.

We first establish that $P(t) \ge 0$ for all $t$. We have:
\[
P(t) = c + 2tb + t^2 a
\]
by expanding the quadratic.

\textbf{Case} $a = 0$: Then $g = 0$, so $b = 0$, and the inequality holds trivially.

\textbf{Case} $a > 0$: We have $P(-b/a) \ge 0$. By simplification:
\[
c - b^2/a \ge 0 \implies b^2 \le ac
\]
Therefore $\abs{b} \le \sqrt{a} \cdot \sqrt{c} = \norm{g}_\pi \cdot \norm{f}_\pi$.
\end{proof}
\end{lemma}

\subsection{Operator Norm}

\begin{definition}[$L^2(\pi)$ Operator Norm]
\label{def:opnorm-pi}
\leanlink{UPAT/Axioms/Geometry.lean}

Given linear map $A : (V \to \mathbb{R}) \to_{\text{lin}} (V \to \mathbb{R})$,
\[
\norm{A}_\pi \coloneqq \inf \{ c \ge 0 \mid \forall f,\, \norm{Af}_\pi \le c \cdot \norm{f}_\pi \}
\]
\end{definition}

\begin{lemma}[Operator Bound]
\label{lem:opnorm-pi-bound}
\leanlink{UPAT/Axioms/Geometry.lean}

\[
\norm{Af}_\pi \le \norm{A}_\pi \cdot \norm{f}_\pi
\]
\begin{proof}
By definition of infimum and the operator norm set.
\end{proof}
\end{lemma}

\begin{lemma}[Submultiplicativity]
\label{lem:opnorm-pi-comp}
\leanlink{UPAT/Axioms/Geometry.lean}

\[
\norm{A \circ B}_\pi \le \norm{A}_\pi \cdot \norm{B}_\pi
\]
\begin{proof}
We have $\norm{(A \circ B)f}_\pi = \norm{A(Bf)}_\pi \le \norm{A}_\pi \cdot \norm{Bf}_\pi \le \norm{A}_\pi \cdot \norm{B}_\pi \cdot \norm{f}_\pi$.
\end{proof}
\end{lemma}

\begin{definition}[Orthogonal Projector onto $\mathbf{1}^\perp$]
\label{def:P-ortho-pi}
\leanlink{UPAT/Axioms/Geometry.lean}

Given $\sum_v \pi(v) = 1$,
\[
P_\perp f \coloneqq f - \ip{f}{\mathbf{1}}_\pi \cdot \mathbf{1}
\]
\end{definition}

% ============================================================================
% PART II: MARKOV BLANKET TOPOLOGY
% Source: UPAT/Topology/Blanket.lean
% ============================================================================

\section{Markov Blanket Topology}
\label{sec:blanket}

\leanlink{UPAT/Topology/Blanket.lean}

\subsection{Blanket Partition Structure}

\begin{structure}[Blanket Partition]
\label{struct:blanket-partition}
\leanlink{UPAT/Topology/Blanket.lean}

A \textbf{Blanket Partition} of $V$ consists of:
\begin{itemize}
    \item $\mu \subseteq V$ \quad (internal states)
    \item $b \subseteq V$ \quad (blanket states)
    \item $\eta \subseteq V$ \quad (external states)
\end{itemize}
satisfying:
\begin{align}
    \mu \cap b &= \emptyset \\
    \mu \cap \eta &= \emptyset \\
    b \cap \eta &= \emptyset \\
    \mu \cup b \cup \eta &= V
\end{align}
\end{structure}

\begin{definition}[Supported On]
\label{def:is-supported-on}
\leanlink{UPAT/Topology/Blanket.lean}

A function $f : V \to \mathbb{R}$ is \textbf{supported on} $S \subseteq V$ iff:
\[
\forall v \notin S,\, f(v) = 0
\]
\end{definition}

\begin{definition}[Internal Function]
\label{def:is-internal-function}
\leanlink{UPAT/Topology/Blanket.lean}

$f$ is an \textbf{internal function} iff $f$ is supported on $\mu$.
\end{definition}

\begin{definition}[External Function]
\label{def:is-external-function}
\leanlink{UPAT/Topology/Blanket.lean}

$g$ is an \textbf{external function} iff $g$ is supported on $\eta$.
\end{definition}

\begin{definition}[Linear Blanket]
\label{def:is-linear-blanket}
\leanlink{UPAT/Topology/Blanket.lean}

A partition satisfies \textbf{Geometric Conditional Independence} iff all internal functions are orthogonal to all external functions:
\[
\forall f \in L^2(\mu),\, \forall g \in L^2(\eta),\quad \ip{f}{g}_\pi = 0
\]
\end{definition}

\begin{definition}[Respects Blanket]
\label{def:respects-blank}
\leanlink{UPAT/Topology/Blanket.lean}

A generator $L : V \times V \to \mathbb{R}$ \textbf{respects} a blanket partition iff:
\[
\forall i \in \mu,\, \forall e \in \eta,\, L_{ie} = 0 \quad \land \quad \forall e \in \eta,\, \forall i \in \mu,\, L_{ei} = 0
\]
\end{definition}

\begin{theorem}[Blanket Orthogonality]
\label{thm:blanket-orthogonality}
\leanlink{UPAT/Topology/Blanket.lean}

Given $L$ respects blanket $B$, and $f$ internal, $g$ external:
\[
\ip{f}{g}_\pi = 0
\]
\begin{proof}
We have $f$ is zero outside $\mu$ and $g$ is zero outside $\eta$. Since $\mu \cap \eta = \emptyset$, for each $v$: if $v \in \mu$ then $g(v) = 0$; if $v \notin \mu$ then $f(v) = 0$. Therefore $f(v) \cdot g(v) = 0$ for all $v$. By $\texttt{Finset.sum\_eq\_zero}$, we obtain $\ip{f}{g}_\pi = 0$.
\end{proof}
\end{theorem}

% ============================================================================
% PART III: INFORMATION-GEOMETRY EQUIVALENCE
% Source: UPAT/Information/Equivalence.lean
% ============================================================================

\section{Information-Geometry Equivalence}
\label{sec:info-geometry}

\leanlink{UPAT/Information/Equivalence.lean}

\subsection{Conditional Mutual Information}

\begin{definition}[Conditional Mutual Information]
\label{def:cmi}
\leanlink{UPAT/Information/Equivalence.lean}

Given precision matrix $P$ and index sets $A, B$:
\[
I(A; B \mid C) \coloneqq \sum_{a \in A} \sum_{b \in B} \abs{P_{ab}}
\]
\end{definition}

\begin{lemma}[CMI Non-negativity]
\label{lem:cmi-nonneg}
\leanlink{UPAT/Information/Equivalence.lean}

\[
I(A; B \mid C) \ge 0
\]
\begin{proof}
By $\texttt{Finset.sum\_nonneg}$ and $\texttt{abs\_nonneg}$.
\end{proof}
\end{lemma}

\subsection{The Gaussian Lemma}

\begin{theorem}[Gaussian CMI-Precision Equivalence]
\label{thm:gaussian-cmi-zero}
\leanlink{UPAT/Information/Equivalence.lean}

\[
I(A; B \mid C) = 0 \iff \forall a \in A,\, \forall b \in B,\, P_{ab} = 0
\]
\begin{proof}
($\Rightarrow$) By $\texttt{Finset.sum\_eq\_zero\_iff\_of\_nonneg}$ applied twice: outer sum zero implies each inner sum zero; inner sum zero implies each $\abs{P_{ab}} = 0$, hence $P_{ab} = 0$.

($\Leftarrow$) If all $P_{ab} = 0$, then $\abs{P_{ab}} = 0$, so each term is zero.
\end{proof}
\end{theorem}

\subsection{Blanket Definitions}

\begin{definition}[Dynamical Blanket]
\label{def:is-dynamical-blanket}
\leanlink{UPAT/Information/Equivalence.lean}

Zero precision between internal and external:
\[
\forall i \in \mu,\, \forall e \in \eta,\, P_{ie} = 0
\]
\end{definition}

\begin{definition}[Information Blanket]
\label{def:is-information-blanket}
\leanlink{UPAT/Information/Equivalence.lean}

Zero CMI between internal and external:
\[
I(\mu; \eta \mid b) = 0
\]
\end{definition}

\begin{theorem}[Dynamical $\Leftrightarrow$ Information Blanket]
\label{thm:dynamical-iff-information}
\leanlink{UPAT/Information/Equivalence.lean}

The dynamical blanket property (transition independence) is equivalent to the information blanket property (conditional independence).
\begin{proof}
By unfolding definitions and applying Theorem~\ref{thm:gaussian-cmi-zero}.
\end{proof}
\end{theorem}

\subsection{The Information Bridge}

\begin{definition}[Information Blanket from Generator]
\label{def:is-information-blanket-v}
\leanlink{UPAT/Information/Equivalence.lean}

\[
\forall i \in \mu,\, \forall e \in \eta,\, L_{ie} = 0
\]
\end{definition}

\begin{definition}[Symmetric Matrix]
\label{def:is-symmetric}
\leanlink{UPAT/Information/Equivalence.lean}

\[
\forall i, j,\, L_{ij} = L_{ji}
\]
\end{definition}

\begin{theorem}[Information Bridge (Forward)]
\label{thm:info-bridge-forward}
\leanlink{UPAT/Information/Equivalence.lean}

If a generator $L$ respects a blanket partition $B$, then it satisfies the information blanket property.
\begin{proof}
Direct from the definition: the first conjunct of the respects-blanket property gives $L_{ie} = 0$.
\end{proof}
\end{theorem}

\begin{theorem}[Symmetric Information Bridge]
\label{thm:symmetric-info-bridge}
\leanlink{UPAT/Information/Equivalence.lean}

Given $L$ symmetric, if $L$ satisfies the information blanket property (i.e., $L_{ie} = 0$ for all $i \in \mu$, $e \in \eta$), then $L$ respects the blanket partition.
\begin{proof}
We construct both conjuncts:
\begin{enumerate}
    \item Internal$\to$External: Given by hypothesis $h$.
    \item External$\to$Internal: Given $e \in \eta$, $i \in \mu$, we have $L_{ei} = L_{ie}$ by symmetry. By hypothesis, $L_{ie} = 0$.
\end{enumerate}
\end{proof}
\end{theorem}

\begin{theorem}[Information-Geometry Equivalence]
\label{thm:info-geometry-equiv}
\leanlink{UPAT/Information/Equivalence.lean}

Given $L$ symmetric, the generator respects the blanket partition if and only if it satisfies the information blanket property:
\[
L_{ie} = 0 \;\forall i \in \mu, e \in \eta \quad \Longleftrightarrow \quad \text{(information blanket)}
\]
\begin{proof}
($\Rightarrow$) By Theorem~\ref{thm:info-bridge-forward}.

($\Leftarrow$) By Theorem~\ref{thm:symmetric-info-bridge}.
\end{proof}
\end{theorem}

\begin{corollary}[Orthogonality $\Leftrightarrow$ Zero Information]
\label{cor:orthog-iff-zero-info}
\leanlink{UPAT/Information/Equivalence.lean}

Given $L$ symmetric, $f$ internal, $g$ external:
If $L$ respects the blanket partition, then $\ip{f}{g}_\pi = 0$. Equivalently, if $L$ satisfies the information blanket property, then $\ip{f}{g}_\pi = 0$.
\begin{proof}
First implication: By Theorem~\ref{thm:blanket-orthogonality}.

Second implication: By Theorem~\ref{thm:symmetric-info-bridge}, $L$ respects the blanket. Then apply Theorem~\ref{thm:blanket-orthogonality}.
\end{proof}
\end{corollary}

% ============================================================================
% PART IV: SPECTRAL GAP THEORY
% Source: UPAT/Stability/Core/Assumptions.lean
% ============================================================================

\section{Spectral Gap Theory}
\label{sec:spectral-gap}

\leanlink{UPAT/Stability/Core/Assumptions.lean}

\subsection{Irreducibility Assumptions}

\begin{structure}[Irreducibility Assumptions]
\label{struct:irreducibility}
\leanlink{UPAT/Stability/Core/Assumptions.lean}

For a lazy, irreducible Markov chain with generator $L$, heat kernel $H$, and stationary distribution $\pi$:
\begin{align}
    \pi &: V \to \mathbb{R} \\
    \forall v,\, &0 < \pi(v) \\
    \sum_v &\pi(v) = 1
\end{align}
\end{structure}

\begin{definition}[Spectral Gap]
\label{def:spectral-gap}
\leanlink{UPAT/Stability/Core/Assumptions.lean}

\[
\gamma \coloneqq \inf \left\{ \frac{\ip{Hv}{v}_\pi}{\ip{v}{v}_\pi} \;\middle|\; v \neq 0,\, \ip{v}{\mathbf{1}}_\pi = 0 \right\}
\]
\end{definition}

\subsection{Spectral Gap Coercivity}

\begin{lemma}[Spectral Gap Coercivity]
\label{lem:spectral-gap-coercivity}
\leanlink{UPAT/Stability/Core/Assumptions.lean}

Given $V$ nontrivial, $H$ self-adjoint and PSD w.r.t.\ $\ip{\cdot}{\cdot}_\pi$, $H \mathbf{1} = 0$, and $\gamma > 0$:
\[
\forall v \perp_\pi \mathbf{1},\quad \ip{Hv}{v}_\pi \ge \gamma \cdot \norm{v}^2_\pi
\]
\begin{proof}
\textbf{Case} $v = 0$: Both sides are zero.

\textbf{Case} $v \neq 0$: We have $\norm{v}^2_\pi > 0$. The Rayleigh quotient $\ip{Hv}{v}_\pi / \norm{v}^2_\pi$ is in the defining set for $\gamma$. By definition of infimum:
\[
\gamma \le \frac{\ip{Hv}{v}_\pi}{\norm{v}^2_\pi}
\]
Multiplying both sides by $\norm{v}^2_\pi > 0$ yields the result.
\end{proof}
\end{lemma}

\subsection{Main Spectral Theorem}

\begin{theorem}[Gap Positive Iff Kernel Equals Span of One]
\label{thm:gap-pos-iff-ker}
\leanlink{UPAT/Stability/Core/Assumptions.lean}

Given $V$ nontrivial, $H$ self-adjoint, PSD, with $H\mathbf{1} = 0$:
\[
\gamma > 0 \iff \ker(H) = \text{span}\{\mathbf{1}\}
\]
\begin{proof}
($\Rightarrow$) Let $u \in \ker(H)$. Decompose $u = v + c\mathbf{1}$ where $v \perp_\pi \mathbf{1}$.

We have $Hv = Hu - cH\mathbf{1} = 0 - 0 = 0$.

If $v \neq 0$, by coercivity: $0 = \ip{Hv}{v}_\pi \ge \gamma \norm{v}^2_\pi > 0$, contradiction.

Therefore $v = 0$, so $u = c\mathbf{1} \in \text{span}\{\mathbf{1}\}$.

($\Leftarrow$) For $u \neq 0$ with $u \perp_\pi \mathbf{1}$:
\begin{itemize}
    \item $u \notin \ker(H)$ since $\ker(H) = \text{span}\{\mathbf{1}\}$ and $u \perp_\pi \mathbf{1}$.
    \item By PSD: $\ip{Hu}{u}_\pi \ge 0$.
    \item If $\ip{Hu}{u}_\pi = 0$, then $Hu = 0$ by the polarization lemma, contradicting $u \notin \ker(H)$.
\end{itemize}
Therefore $\ip{Hu}{u}_\pi > 0$ for all nonzero $u \perp_\pi \mathbf{1}$.

By compactness of the normalized set $\{v : \norm{v}^2_\pi = 1,\, v \perp_\pi \mathbf{1}\}$ in finite dimension, the continuous function $v \mapsto \ip{Hv}{v}_\pi$ achieves a positive minimum. Hence $\gamma > 0$.
\end{proof}
\end{theorem}

% ============================================================================
% END OF FORMALISM
% ============================================================================
