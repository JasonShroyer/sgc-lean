% ============================================================================
% VITALITY: Doob-Meyer Decomposition of the Surprise Process
% ============================================================================
% Literal translation from:
%   - src/UPAT/Vitality/DoobMeyer.lean
% ============================================================================

\section{Vitality: Doob-Meyer Decomposition}
\label{sec:vitality}

% ----------------------------------------------------------------------------
\subsection{Stochastic Assumptions}
% ----------------------------------------------------------------------------

\begin{definition}[Stochastic Matrix]
\label{def:is-stochastic}
\leanlink{UPAT/Vitality/DoobMeyer.lean}

A matrix $P : V \times V \to \mathbb{R}$ is \textbf{stochastic} iff:
\[
(\forall x, y,\; 0 \le P_{xy}) \;\land\; (\forall x,\; \textstyle\sum_y P_{xy} = 1)
\]
\end{definition}

\begin{definition}[Stationary Distribution]
\label{def:is-stationary}
\leanlink{UPAT/Vitality/DoobMeyer.lean}

$\pi$ is a \textbf{stationary distribution} for $P$ iff $\pi P = \pi$ (as row vector):
\[
\forall y,\quad \sum_x \pi_x \cdot P_{xy} = \pi_y
\]
\end{definition}

\begin{definition}[Detailed Balance]
\label{def:is-detailed-balance}
\leanlink{UPAT/Vitality/DoobMeyer.lean}

$P$ satisfies \textbf{detailed balance} with respect to $\pi$ iff:
\[
\forall x, y,\quad \pi_x \cdot P_{xy} = \pi_y \cdot P_{yx}
\]
This implies reversibility.
\end{definition}

% ----------------------------------------------------------------------------
\subsection{The Surprise Potential}
% ----------------------------------------------------------------------------

\begin{definition}[Surprise Potential]
\label{def:surprise-potential}
\leanlink{UPAT/Vitality/DoobMeyer.lean}

Given $\pi : V \to \mathbb{R}^+$ with $\forall x,\, \pi_x > 0$, the \textbf{Surprise Potential} is:
\[
\Phi(x) \coloneqq -\log \pi_x
\]
This measures the ``unexpectedness'' of state $x$. Lower probability states have higher surprise.
\end{definition}

% ----------------------------------------------------------------------------
\subsection{Conditional Expectation (Discrete)}
% ----------------------------------------------------------------------------

\begin{definition}[Conditional Expectation]
\label{def:cond-exp}
\leanlink{UPAT/Vitality/DoobMeyer.lean}

The \textbf{conditional expectation} of $f(X')$ given $X = x$, using transition matrix $P$:
\[
\mathbb{E}[f(X') \mid X = x] \coloneqq \sum_y P_{xy} \cdot f(y)
\]
This is the discrete, Finset-based definition avoiding measure theory.
\end{definition}

Using the linearity of conditional expectation and the non-negativity of surprise (Appendix~\ref{sec:aux:probability}), we derive the fundamental decomposition of thermodynamic evolution.

% ----------------------------------------------------------------------------
\subsection{The Doob Decomposition}
% ----------------------------------------------------------------------------

\begin{definition}[Predictable Increment]
\label{def:predictable-increment}
\leanlink{UPAT/Vitality/DoobMeyer.lean}

The \textbf{one-step predictable increment} of a potential $\Phi$:
\[
\Delta A(x) \coloneqq \mathbb{E}[\Phi(X') \mid X = x] - \Phi(x)
\]
This is the expected change in $\Phi$, which is predictable given $X = x$.
\end{definition}

\begin{definition}[Martingale Increment]
\label{def:martingale-increment}
\leanlink{UPAT/Vitality/DoobMeyer.lean}

The \textbf{martingale increment} (unpredictable part) for transition $x \to y$:
\[
\Delta M(x, y) \coloneqq \Phi(y) - \mathbb{E}[\Phi(X') \mid X = x]
\]
This is the ``surprise'' beyond what was expected.
\end{definition}

\begin{theorem}[Doob Decomposition Identity]
\label{thm:doob-decomposition}
\leanlink{UPAT/Vitality/DoobMeyer.lean}

The actual change equals predictable + martingale:
\[
\Phi(y) - \Phi(x) = \Delta A(x) + \Delta M(x, y)
\]

\begin{proof}
By substitution:
\begin{align*}
\Delta A(x) + \Delta M(x, y) 
&= (\mathbb{E}[\Phi \mid x] - \Phi(x)) + (\Phi(y) - \mathbb{E}[\Phi \mid x]) \\
&= \Phi(y) - \Phi(x)
\end{align*}
\end{proof}
\end{theorem}

\begin{theorem}[Martingale Increment Has Zero Expectation]
\label{thm:martingale-increment-zero}
\leanlink{UPAT/Vitality/DoobMeyer.lean}

The defining property of a martingale:
\[
\mathbb{E}[\Delta M(X, X') \mid X = x] = 0
\]

\begin{proof}
\begin{align*}
\mathbb{E}[\Delta M \mid x] 
&= \sum_y P_{xy} \cdot (\Phi(y) - \mathbb{E}[\Phi \mid x]) \\
&= \sum_y P_{xy} \cdot \Phi(y) - \mathbb{E}[\Phi \mid x] \cdot \sum_y P_{xy} \\
&= \mathbb{E}[\Phi \mid x] - \mathbb{E}[\Phi \mid x] \cdot 1 = 0
\end{align*}
\end{proof}
\end{theorem}

% ----------------------------------------------------------------------------
\subsection{Martingale Properties}
% ----------------------------------------------------------------------------

\begin{definition}[Supermartingale]
\label{def:supermartingale}
\leanlink{UPAT/Vitality/DoobMeyer.lean}

$\Phi$ is a \textbf{supermartingale} under $P$ iff:
\[
\forall x,\quad \mathbb{E}[\Phi(X') \mid X = x] \le \Phi(x)
\]
Equivalently, the predictable increment is non-positive.
For surprise, this means expected surprise \emph{decreases} (consolidation).
\end{definition}

\begin{definition}[Submartingale]
\label{def:submartingale}
\leanlink{UPAT/Vitality/DoobMeyer.lean}

$\Phi$ is a \textbf{submartingale} under $P$ iff:
\[
\forall x,\quad \Phi(x) \le \mathbb{E}[\Phi(X') \mid X = x]
\]
For surprise, this means expected surprise \emph{increases} (dissolution).
\end{definition}

\begin{definition}[Martingale]
\label{def:martingale}
\leanlink{UPAT/Vitality/DoobMeyer.lean}

$\Phi$ is a \textbf{martingale} under $P$ iff:
\[
\forall x,\quad \mathbb{E}[\Phi(X') \mid X = x] = \Phi(x)
\]
For surprise, this is the equilibrium condition.
\end{definition}

\begin{theorem}[Martingale Characterization]
\label{thm:martingale-iff-super-sub}
\leanlink{UPAT/Vitality/DoobMeyer.lean}

A potential $\Phi$ is a martingale if and only if it is both a supermartingale and a submartingale:
\[
\mathbb{E}[\Phi(X') \mid X] = \Phi(X) \quad \Longleftrightarrow \quad 
\mathbb{E}[\Phi(X') \mid X] \le \Phi(X) \;\wedge\; \mathbb{E}[\Phi(X') \mid X] \ge \Phi(X)
\]

\begin{proof}
$(\Rightarrow)$ Equality implies both $\le$ and $\ge$.
$(\Leftarrow)$ By \texttt{le\_antisymm}.
\end{proof}
\end{theorem}

\begin{theorem}[Supermartingale Drift is Non-Positive]
\label{thm:supermartingale-drift-nonpos}
\leanlink{UPAT/Vitality/DoobMeyer.lean}

For a supermartingale:
\[
\forall x,\quad \Delta A(x) \le 0
\]

\begin{proof}
$\Delta A(x) = \mathbb{E}[\Phi \mid x] - \Phi(x) \le 0$ by definition of supermartingale.
\end{proof}
\end{theorem}

\begin{theorem}[Submartingale Drift is Non-Negative]
\label{thm:submartingale-drift-nonneg}
\leanlink{UPAT/Vitality/DoobMeyer.lean}

For a submartingale:
\[
\forall x,\quad \Delta A(x) \ge 0
\]

\begin{proof}
$\Delta A(x) = \mathbb{E}[\Phi \mid x] - \Phi(x) \ge 0$ by definition of submartingale.
\end{proof}
\end{theorem}

% ----------------------------------------------------------------------------
\subsection{Free Energy Minimization Implies Consolidation}
% ----------------------------------------------------------------------------

\begin{theorem}[Contraction Implies Supermartingale]
\label{thm:contraction-supermartingale}
\leanlink{UPAT/Vitality/DoobMeyer.lean}

If the system is ``contracting'' toward the stationary distribution 
(relative entropy decreasing), then surprise is a supermartingale:
\[
\forall x,\; \mathbb{E}[\Phi(X') \mid X = x] \le \Phi(x)
\]
That is, $\Phi$ satisfies the supermartingale property under $P$.

This formalizes the \textbf{Rate of Consolidation}: systems naturally evolve
toward lower surprise (higher probability) states.

\begin{proof}
Direct from the hypothesis.
\end{proof}
\end{theorem}

% ----------------------------------------------------------------------------
\subsection{Bridge to Blankets: Leakage Variance}
% ----------------------------------------------------------------------------

\begin{definition}[Blanket Leakage]
\label{def:blanket-leakage}
\leanlink{UPAT/Vitality/DoobMeyer.lean}

For $x \in \mu$ (internal), the \textbf{leakage} at the blanket is:
\[
\mathrm{Leakage}(x) \coloneqq \sum_{y \in \eta} P_{xy} \cdot \Delta M(x, y)
\]
This measures unpredictable information flow across the blanket boundary.
\end{definition}

\begin{theorem}[Bottleneck Bounds Leakage]
\label{thm:bottleneck-leakage}
\leanlink{UPAT/Vitality/DoobMeyer.lean}

If $P$ respects the blanket partition $B$, then for $x \in \mu$:
\[
\mathrm{Leakage}(x) = 0
\]

\begin{proof}
If $P$ respects the blanket, there are no direct internal $\to$ external transitions.
By \texttt{RespectsBlank}, $P_{xy} = 0$ for $x \in \mu$, $y \in \eta$.
Therefore each term in the sum is zero.
\end{proof}
\end{theorem}

% ----------------------------------------------------------------------------
\subsection{Doob Structure Theorem (Summary)}
% ----------------------------------------------------------------------------

\begin{theorem}[Doob Structure]
\label{thm:doob-structure}
\leanlink{UPAT/Vitality/DoobMeyer.lean}

For any Markov chain with transition matrix $P$ and stationary distribution $\pi$:

\begin{enumerate}
    \item Surprise $\Phi = -\log \pi$ decomposes as $\Delta\Phi = \Delta A + \Delta M$
    \item $M_n$ is a martingale: $\mathbb{E}[\Delta M \mid \mathcal{F}_n] = 0$
    \item $A_n$ is predictable: $\mathcal{F}_n$-measurable
    \item At detailed balance equilibrium: $\Delta A = 0$
    \item Away from equilibrium: $\Delta A < 0$ (contraction)
    \item Blanket structure bounds cross-boundary leakage
\end{enumerate}

\begin{proof}
Combines Theorems~\ref{thm:doob-decomposition}, \ref{thm:martingale-increment-zero}, 
\ref{thm:contraction-supermartingale}, and \ref{thm:bottleneck-leakage}.
\end{proof}
\end{theorem}

% ============================================================================
% END VITALITY
% ============================================================================
